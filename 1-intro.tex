\begin{frame}{Problema Exemplo}
    \defproblemaOPT
        {Conjunto Independente Máximo (CI)}
        {Grafo $G = (V, E)$.\pause}
        {$S \subseteq V$ tal que $|E(G[S])|=0$, com $|S|$ máximo.}
\end{frame}

\begin{frame}{Dificuldade}
    \begin{thm}[\cite{Ga79}]
        O problema do Conjunto Independente Máximo é NP-difícil, mesmo em grafos cúbicos e planares.
    \end{thm}
\end{frame}

\begin{frame}{Limitante de Aproximação}
    \begin{thm}[\cite{Hstad99}]
        Seja $n$ o número de vértices de um grafo, e $\varepsilon > 0$. \medskip \pause

        Não existe aproximação com fator $O(n^{\varepsilon - 1})$ para CI, a menos que $P = NP$.
    \end{thm}
\end{frame}

\begin{frame}{Algoritmo para Árvores}
    \begin{thm}[folclore]
        Seja $T$ uma árvore. É possível computar um conjunto independente máximo de $T$ em tempo polinomial.
    \end{thm}
\end{frame}
